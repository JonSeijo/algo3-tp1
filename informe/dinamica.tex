\subsection{Solución topdown}

Aca va la explicacion de la solucion y el por que el algoritmo es correcto

\subsection{Pseudocódigo topdown}

\begin{algorithm}[H]
\begin{algorithmic}
\Procedure{Resolver topdown}{secuencia(int) $A$}
  \State Matriz3 $DP \gets$ Matriz3($n$, $-1$) \Comment Matriz de 3 dimensiones, llena con $-1$
    \For {$ultRojo \in [0..n]$}
        \For {$ultAzul \in [0..n]$}
            \State $minSinPintar \gets$ Min($minSinPintar$, Solución($n-1$, $ultRojo$, $ultAzul$))
        \EndFor
    \EndFor
    \State return $minSinPintar$
\EndProcedure
\end{algorithmic}
\end{algorithm} 


\begin{algorithm}[H]
\begin{algorithmic}
\Procedure{Solución}{int $actual$, int $ultRojo$, int $ultAzul$} %\Comment($ur$: último rojo;  $ua$: último azul)
    \If {$actual = -1$}
        return $0$
    \EndIf
    \If {DP[$actual$][$ultRojo$][$ultAzul$] $\neq -1$}
        return DP[$actual$][$ultRojo$][$ultAzul$]
    \EndIf \\

    \State $minRojo \gets$ TopdownCasoRojo($actual$, $ultRojo$, $ultAzul$)
    \State $minAzul \gets$ TopdownCasoAzul($actual$, $ultRojo$, $ultAzul$)
    \State $minSinPintar \gets$ TopdownCasoSinPintar($actual$, $ultRojo$, $ultAzul$) \\

    \State return Min($minRojo$, $minAzul$, $minSinPintar$)
\EndProcedure
\end{algorithmic}
\end{algorithm} 


\begin{algorithm}[H]
\begin{algorithmic}
\Procedure{TopdownCasoRojo}{int $actual$, int $ultRojo$, int $ultAzul$} %\Comment($ur$: último rojo;  $ua$: último azul)
    \Comment Si no hay rojo o si el actual es azul, entonces no puedo considerar que el actual sea rojo
    \If {$(ultRojo = n)  \lor (actual = ultAzul)$}
        \State return $\infty$
    
    \Else \Comment Si soy el último rojo, ó si puedo serlo porque cumplo la propiedad:
        \If {($actual = ultRojo$) $\lor$ ($ i < ultRojo \land A[i] < A[ultRojo]$)}
            \State return Solución($actual - 1$, $actual$, $ultAzul$)
        \Else 
            \State return $\infty$
        \EndIf 
    \EndIf
\EndProcedure
\end{algorithmic}
\end{algorithm}


\begin{algorithm}[H]
\begin{algorithmic}
\Procedure{TopdownCasoAzul}{int $actual$, int $ultRojo$, int $ultAzul$} \\ %\Comment($ur$: último rojo;  $ua$: último azul)
    \Comment Si no hay azul o si el actual es rojo, entonces no puedo considerar que el actual sea azul
    \If {$(ultAzul = n)  \lor (actual = ultRojo)$}
        \State return $\infty$
    
    \Else \Comment Si soy el último azul, ó si puedo serlo porque cumplo la propiedad:
        \If {($actual = ultAzul$) $\lor$ ($ i < ultAzul \land A[i] > A[ultAzul]$)}
            \State return Solución($actual - 1$, $ultAzul$, $actual$)
        \Else 
            \State return $\infty$
        \EndIf 
    \EndIf
\EndProcedure
\end{algorithmic}
\end{algorithm} 


\begin{algorithm}[H]
\begin{algorithmic}
\Procedure{TopdownCasoSinPintar}{int $actual$, int $ultRojo$, int $ultAzul$} \\
    \Comment No puede pasar que el actual sea rojo o azul, pero que lo quiera dejar sin pintar
    \If {$(actual = ultRojo)  \lor (actual = ultAzul)$}
        \State return $\infty$
    \Else 
        \State return $1 + $  Solución($actual - 1$, $ultRojo$, $ultAzul$)
    \EndIf
\EndProcedure
\end{algorithmic}
\end{algorithm}


\subsection{Complejidad}

Aca va la justificacion de por que el algoritmo topdown es $n^3$, consultar mis notas para orientacion
