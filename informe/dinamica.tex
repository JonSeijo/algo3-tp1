\subsection{Solución topdown}

Generar todas las combinaciones es muy caro computacionalmente. Se puede conseguir algo mejor si pensamos el problema desde otro ángulo. \\

Supongamos que para todo ($i$, $j$) ya tenemos el valor mínimo de elementos sin pintar dado que el $i$-ésimo es el último rojo y el $j$-ésimo es el último azul. De este modo, lo único que tendríamos que hacer es tomar el mínimo de todas las combinaciones de los últimos rojos y azules. Esto es así porque de todas las posibles soluciones para un último rojo y último azul fijos, hay una combinación que es óptima, y es ésta la que nos interesa obtener. \\

Entonces nuestro problema se reduce a: suponiedo que la secuencia tiene el último rojo y el último azul en posiciones fijas, ¿cuál es la mínima cantidad de elementos sin pintar que podemos obtener? \\

Llamo Solución($i$, $ur$, $ua$) a la función que devuelve la mínima cantidad sin pintar hasta $A$[$i$] siendo que el elemento en $ur$ es el último rojo y el elemento en $ua$ es el último azul. (Si $ur$ o $ua$ es $n$, represento que no hay rojo o azul) \\

%Cuando $i$ es menor a cero, la respuesta es cero

Analizo Solución($i$, $ur$, $ua$). Para el elemento en la posicion $i$ (dados un $ur$ y un $ua$) existen 3 casos \\

- \textbf{No lo pinto}: En este caso, la cantidad de elementos sin pintar \textbf{aumenta en uno} con respecto al mínimo hasta $i-1$. Es decir, la solución es igual a $1 + $Solución($i-1$, $ur$, $ua$) \\

- \textbf{Pinto $i$ de rojo}: 
\begin{enumerate}
\item Si estoy en la rama donde no hay rojo ($ultRojo = n$), o si estoy en la rama donde el último azul era $i$, entonces no puede ser que $i$ sea rojo, por lo que no hay solución. (Devuelvo $\infty$)
\item Si $i$ es el último rojo, o si es posible que $i$ sea color rojo (pues no rompe la propiedad) entonces la solución es la misma que la solución hasta $i-1$ siendo que $i$ es el último rojo. Es decir, es igual a Solución($i-1$, $i$, $ua$)
\item Si no era posible que $i$ sea rojo, entonces no hay solucion para este caso. (Devuelvo $\infty$)
\end{enumerate} 

- \textbf{Pinto $i$ de azul}:  Análogo al caso de pintar con rojo  \\

Al final de la funcion Solución($i$, $ur$, $ua$) devuelvo el mínimo de los tres casos. El valor de la solución para esos parámetros lo guardo en una matriz para poder acceder a él y no recalcularlo. \\

Para resolver el problema, tomo el mínimo de todas las soluciones posibles (para todo $ur$ y $ua$) hasta el n-ésimo elemento.

\subsection{Pseudocódigo topdown}

\begin{algorithm}[H]
\begin{algorithmic}
\Procedure{Resolver topdown}{secuencia(int) $A$}
  \State Matriz3 $DP \gets$ Matriz3($n$, $-1$) \Comment Matriz de 3 dimensiones, llena con $-1$
    \For {$ultRojo \in [0..n]$}
        \For {$ultAzul \in [0..n]$}
            \State $minSinPintar \gets$ Min($minSinPintar$, Solución($n-1$, $ultRojo$, $ultAzul$))
        \EndFor
    \EndFor
    \State return $minSinPintar$
\EndProcedure
\end{algorithmic}
\end{algorithm} 


\begin{algorithm}[H]
\begin{algorithmic}
\Procedure{Solución}{int $actual$, int $ultRojo$, int $ultAzul$} %\Comment($ur$: último rojo;  $ua$: último azul)
    \If {$actual = -1$}
        return $0$
    \EndIf
    \If {DP[$actual$][$ultRojo$][$ultAzul$] $\neq -1$}
        return DP[$actual$][$ultRojo$][$ultAzul$]
    \EndIf \\

    \State $minRojo \gets$ TopdownCasoRojo($actual$, $ultRojo$, $ultAzul$)
    \State $minAzul \gets$ TopdownCasoAzul($actual$, $ultRojo$, $ultAzul$)
    \State $minSinPintar \gets$ TopdownCasoSinPintar($actual$, $ultRojo$, $ultAzul$) \\

    \State return Min($minRojo$, $minAzul$, $minSinPintar$)
\EndProcedure
\end{algorithmic}
\end{algorithm} 


\begin{algorithm}[H]
\begin{algorithmic}
\Procedure{TopdownCasoRojo}{int $actual$, int $ultRojo$, int $ultAzul$} \\ 
    \Comment Si no hay rojo o si el actual es azul, entonces no puedo considerar que el actual sea rojo
    \If {$(ultRojo = n)  \lor (actual = ultAzul)$}
        \State return $\infty$
    
    \Else \Comment Si soy el último rojo, ó si puedo serlo porque cumplo la propiedad:
        \If {($actual = ultRojo$) $\lor$ ($ i < ultRojo \land A[i] < A[ultRojo]$)}
            \State return Solución($actual - 1$, $actual$, $ultAzul$)
        \Else 
            \State return $\infty$
        \EndIf 
    \EndIf
\EndProcedure
\end{algorithmic}
\end{algorithm}


\begin{algorithm}[H]
\begin{algorithmic}
\Procedure{TopdownCasoAzul}{int $actual$, int $ultRojo$, int $ultAzul$} \\ 
    \Comment Si no hay azul o si el actual es rojo, entonces no puedo considerar que el actual sea azul
    \If {$(ultAzul = n)  \lor (actual = ultRojo)$}
        \State return $\infty$
    
    \Else \Comment Si soy el último azul, ó si puedo serlo porque cumplo la propiedad:
        \If {($actual = ultAzul$) $\lor$ ($ i < ultAzul \land A[i] > A[ultAzul]$)}
            \State return Solución($actual - 1$, $ultAzul$, $actual$)
        \Else 
            \State return $\infty$
        \EndIf 
    \EndIf
\EndProcedure
\end{algorithmic}
\end{algorithm} 


\begin{algorithm}[H]
\begin{algorithmic}
\Procedure{TopdownCasoSinPintar}{int $actual$, int $ultRojo$, int $ultAzul$} \\
    \Comment No puede pasar que el actual sea rojo o azul, pero que lo quiera dejar sin pintar
    \If {$(actual = ultRojo)  \lor (actual = ultAzul)$}
        \State return $\infty$
    \Else 
        \State return $1 + $  Solución($actual - 1$, $ultRojo$, $ultAzul$)
    \EndIf
\EndProcedure
\end{algorithmic}
\end{algorithm}


\subsection{Complejidad}

Aca va la justificacion de por que el algoritmo topdown es $n^3$, consultar mis notas para orientacion
