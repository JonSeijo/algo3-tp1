Dada una secuencia de números $A$, se quiere pintar algunos de sus números de color rojo o azul. Dicho de otro modo, cada número de la secuencia puede ser pintado de color rojo, azul o de ningun color. \\

Una aclaración importante es que los elementos de $A$ no pueden modificarse, ni tampoco cambiarse su orden inicial. Lo unico que puede hacerse con ellos es colorearlos (o no).\\

Pintando $A$ de alguna forma se obtiene una secuencia de colores, pero no todas las secuencias de colores que pueden obtenerse son válidas.\\

Decimos que una secuencia de colores es válida si se cumplen las siguientes condiciones:
\begin{enumerate}
\item Todos los elementos de color \textcolor{red}{rojo} están ordenados por valor de forma \underline{estrictamente creciente}
\item Todos los elementos de color \textcolor{blue}{azul} están ordenados por valor de forma \underline{estrictamente decreciente}
\end{enumerate}

Las secuencias de colores válidas pueden tener diferentes cantidades de elementos sin pintar. \\
El objetivo del problema es encontrar la mínima cantidad de elementos sin pintar de todas las secuencias válidas que pueden formarse a partir de $A$. \\

% Más formalmente, \\
% Sea $A$ una secuencia tal que \\
% \centering {
% $(\forall e \in A) ($ color($e$) $\in$ $\{$Rojo, Azul, Ninguno$\}$) \\
% }
